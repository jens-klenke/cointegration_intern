\documentclass[11pt,a4paper]{article}
\usepackage{lmodern}

\usepackage{amssymb,amsmath}
\usepackage{ifxetex,ifluatex}
\usepackage{fixltx2e} % provides \textsubscript
\ifnum 0\ifxetex 1\fi\ifluatex 1\fi=0 % if pdftex
  \usepackage[T1]{fontenc}
  \usepackage[utf8]{inputenc}
\else % if luatex or xelatex
  \ifxetex
    \usepackage{mathspec}
    \usepackage{xltxtra,xunicode}
  \else
    \usepackage{fontspec}
  \fi
  \defaultfontfeatures{Mapping=tex-text,Scale=MatchLowercase}
  \newcommand{\euro}{€}
\fi
% use upquote if available, for straight quotes in verbatim environments
\IfFileExists{upquote.sty}{\usepackage{upquote}}{}
% use microtype if available
\IfFileExists{microtype.sty}{%
\usepackage{microtype}
\UseMicrotypeSet[protrusion]{basicmath} % disable protrusion for tt fonts
}{}
\usepackage[lmargin=5cm,rmargin=2.5cm,tmargin=2.5cm,bmargin=2.5cm]{geometry}

% Figure Placement:
\usepackage{float}
\let\origfigure\figure
\let\endorigfigure\endfigure
\renewenvironment{figure}[1][2] {
    \expandafter\origfigure\expandafter[H]
} {
    \endorigfigure
}

%% citation setup

\usepackage{csquotes}

\usepackage[backend=biber, maxbibnames = 99, style = apa]{biblatex}
\setlength\bibitemsep{1.5\itemsep}
\bibliography{references.bib}
\usepackage{color}
\usepackage{fancyvrb}
\newcommand{\VerbBar}{|}
\newcommand{\VERB}{\Verb[commandchars=\\\{\}]}
\DefineVerbatimEnvironment{Highlighting}{Verbatim}{commandchars=\\\{\}}
% Add ',fontsize=\small' for more characters per line
\usepackage{framed}
\definecolor{shadecolor}{RGB}{248,248,248}
\newenvironment{Shaded}{\begin{snugshade}}{\end{snugshade}}
\newcommand{\AlertTok}[1]{\textcolor[rgb]{0.94,0.16,0.16}{#1}}
\newcommand{\AnnotationTok}[1]{\textcolor[rgb]{0.56,0.35,0.01}{\textbf{\textit{#1}}}}
\newcommand{\AttributeTok}[1]{\textcolor[rgb]{0.77,0.63,0.00}{#1}}
\newcommand{\BaseNTok}[1]{\textcolor[rgb]{0.00,0.00,0.81}{#1}}
\newcommand{\BuiltInTok}[1]{#1}
\newcommand{\CharTok}[1]{\textcolor[rgb]{0.31,0.60,0.02}{#1}}
\newcommand{\CommentTok}[1]{\textcolor[rgb]{0.56,0.35,0.01}{\textit{#1}}}
\newcommand{\CommentVarTok}[1]{\textcolor[rgb]{0.56,0.35,0.01}{\textbf{\textit{#1}}}}
\newcommand{\ConstantTok}[1]{\textcolor[rgb]{0.00,0.00,0.00}{#1}}
\newcommand{\ControlFlowTok}[1]{\textcolor[rgb]{0.13,0.29,0.53}{\textbf{#1}}}
\newcommand{\DataTypeTok}[1]{\textcolor[rgb]{0.13,0.29,0.53}{#1}}
\newcommand{\DecValTok}[1]{\textcolor[rgb]{0.00,0.00,0.81}{#1}}
\newcommand{\DocumentationTok}[1]{\textcolor[rgb]{0.56,0.35,0.01}{\textbf{\textit{#1}}}}
\newcommand{\ErrorTok}[1]{\textcolor[rgb]{0.64,0.00,0.00}{\textbf{#1}}}
\newcommand{\ExtensionTok}[1]{#1}
\newcommand{\FloatTok}[1]{\textcolor[rgb]{0.00,0.00,0.81}{#1}}
\newcommand{\FunctionTok}[1]{\textcolor[rgb]{0.00,0.00,0.00}{#1}}
\newcommand{\ImportTok}[1]{#1}
\newcommand{\InformationTok}[1]{\textcolor[rgb]{0.56,0.35,0.01}{\textbf{\textit{#1}}}}
\newcommand{\KeywordTok}[1]{\textcolor[rgb]{0.13,0.29,0.53}{\textbf{#1}}}
\newcommand{\NormalTok}[1]{#1}
\newcommand{\OperatorTok}[1]{\textcolor[rgb]{0.81,0.36,0.00}{\textbf{#1}}}
\newcommand{\OtherTok}[1]{\textcolor[rgb]{0.56,0.35,0.01}{#1}}
\newcommand{\PreprocessorTok}[1]{\textcolor[rgb]{0.56,0.35,0.01}{\textit{#1}}}
\newcommand{\RegionMarkerTok}[1]{#1}
\newcommand{\SpecialCharTok}[1]{\textcolor[rgb]{0.00,0.00,0.00}{#1}}
\newcommand{\SpecialStringTok}[1]{\textcolor[rgb]{0.31,0.60,0.02}{#1}}
\newcommand{\StringTok}[1]{\textcolor[rgb]{0.31,0.60,0.02}{#1}}
\newcommand{\VariableTok}[1]{\textcolor[rgb]{0.00,0.00,0.00}{#1}}
\newcommand{\VerbatimStringTok}[1]{\textcolor[rgb]{0.31,0.60,0.02}{#1}}
\newcommand{\WarningTok}[1]{\textcolor[rgb]{0.56,0.35,0.01}{\textbf{\textit{#1}}}}
\usepackage{graphicx}
\makeatletter
\def\maxwidth{\ifdim\Gin@nat@width>\linewidth\linewidth\else\Gin@nat@width\fi}
\def\maxheight{\ifdim\Gin@nat@height>\textheight\textheight\else\Gin@nat@height\fi}
\makeatother
% Scale images if necessary, so that they will not overflow the page
% margins by default, and it is still possible to overwrite the defaults
% using explicit options in \includegraphics[width, height, ...]{}
\setkeys{Gin}{width=\maxwidth,height=\maxheight,keepaspectratio}
\ifxetex
  \usepackage[setpagesize=false, % page size defined by xetex
              unicode=false, % unicode breaks when used with xetex
              xetex]{hyperref}
\else
  \usepackage[unicode=true]{hyperref}
\fi
\hypersetup{breaklinks=true,
            bookmarks=true,
            pdfauthor={Name},
            pdftitle={Title},
            colorlinks=true,
            citecolor=blue,
            urlcolor=blue,
            linkcolor=magenta,
            pdfborder={0 0 0}}
\urlstyle{same}  % don't use monospace font for urls
\setlength{\parindent}{0pt}
\setlength{\parskip}{6pt plus 2pt minus 1pt}
\setlength{\emergencystretch}{3em}  % prevent overfull lines
\setcounter{secnumdepth}{5}

%%% Use protect on footnotes to avoid problems with footnotes in titles
\let\rmarkdownfootnote\footnote%
\def\footnote{\protect\rmarkdownfootnote}

%%% Change title format to be more compact
\usepackage{titling}

% Create subtitle command for use in maketitle
\newcommand{\subtitle}[1]{
  \posttitle{
    \begin{center}\large#1\end{center}
    }
}

\setlength{\droptitle}{-2em}
  \title{Title}
  \pretitle{\vspace{\droptitle}\centering\huge}
  \posttitle{\par}
\subtitle{Subtitle}
  \author{Name}
  \preauthor{\centering\large\emph}
  \postauthor{\par}
  \predate{\centering\large\emph}
  \postdate{\par}
  \date{today}


%% linespread settings

\usepackage{setspace}

\onehalfspacing

% Language Setup

\usepackage{ifthen}
\usepackage{iflang}
\usepackage[super]{nth}
\usepackage[ngerman, english]{babel}

\begin{document}

\selectlanguage{english}


%\maketitle

\newgeometry{left=2cm,right=1cm,bottom=2cm,top=2cm}

\begin{titlepage}
  \noindent\begin{minipage}{0.6\textwidth}
	  \IfLanguageName{english}{University of Duisburg-Essen}{Universität Duisburg-Essen}\\
	  \IfLanguageName{english}{Faculty of Business Administration and Economics}{Fakultät für Wirtschaftswissensschaften}\\
	  \IfLanguageName{english}{Chair of Econometrics}{Lehrstuhl für Ökonometrie}\\
  \end{minipage}
	\begin{minipage}{0.4\textwidth}
	  \begin{flushright}
  	  \vspace{-0.5cm}
      \IfLanguageName{english}{\includegraphics*[width=5cm]{Includes/duelogo_en.png}}{\includegraphics*[width=5cm]{Includes/duelogo_de.png}}
	  \end{flushright}
	\end{minipage}
  \\
  \vspace{1.5cm}
  \begin{center}
  \huge{Title}\\
  \vspace{.25cm}
  \Large{Subtitle}\\
  \vspace{0.5cm}
  \large{Type of Paper}\\
  \vspace{1cm}
  \large{
  \IfLanguageName{english}{Submitted to the Faculty of \\ Business Administration and Economics \\at the \\University of Duisburg-Essen}{Vorgelegt der \\Fakultät für Wirtschaftswissenschaften der \\ Universität Duisburg-Essen}\\}
  \vspace{0.75cm}
  \large{\IfLanguageName{english}{from:}{von:}}\\
  \vspace{0.5cm}
  Name\\
  \end{center}
  \vspace{4cm}

  \noindent\begin{minipage}[t]{0.5\textwidth}
  \IfLanguageName{english}{Matriculation Number:}{Matrikelnummer}
  \end{minipage}
  \begin{minipage}[t]{0.7\textwidth}
  \hspace{1cm}Matriculation Number
  \end{minipage}

  \noindent\begin{minipage}[t]{0.5\textwidth}
  \IfLanguageName{english}{Study Path:}{Studienfach:}
  \end{minipage}
  \begin{minipage}[t]{0.7\textwidth}
  \hspace{1cm}Study Path
  \end{minipage}

  \noindent\begin{minipage}[t]{0.5\textwidth}
  \IfLanguageName{english}{Reviewer:}{Erstgutachter:}
  \end{minipage}
  \begin{minipage}[t]{0.7\textwidth}
  \hspace{1cm}Prof.~Dr.~Christoph Hanck
  \end{minipage}

  \noindent\begin{minipage}[t]{0.5\textwidth}
  \IfLanguageName{english}{Secondary Reviewer:}{Zweitgutachter}
  \end{minipage}
  \begin{minipage}[t]{0.7\textwidth}
  \hspace{1cm}Prof.~Dr.~Andreas Behr
  \end{minipage}

  \noindent\begin{minipage}[t]{0.5\textwidth}
  Semester:
  \end{minipage}
  \begin{minipage}[t]{0.7\textwidth}
  \hspace{1cm}\IfLanguageName{english}{\nth{1} Semester}{1. Fachsemester}
  \end{minipage}

  \noindent\begin{minipage}[t]{0.5\textwidth}
  \IfLanguageName{english}{Graduation (est.):}{Vsl. Studienabschluss:}
  \end{minipage}
  \begin{minipage}[t]{0.7\textwidth}
  \hspace{1cm}Summer Term 2019
  \end{minipage}

  \noindent\begin{minipage}[t]{0.5\textwidth}
  \IfLanguageName{english}{Deadline:}{Abgabefrist:}
  \end{minipage}
  \begin{minipage}[t]{0.7\textwidth}
  \hspace{1cm}Deadline
  \end{minipage}

\end{titlepage}

% Ends the declared geometry for the titlepage
\restoregeometry


\pagenumbering{Roman} 
{
\hypersetup{linkcolor=black}
\setcounter{tocdepth}{3}
\tableofcontents
}
\newpage
\listoftables
\newpage
\listoffigures
\newpage
\pagenumbering{arabic} 
\hypertarget{introduction}{%
\section{Introduction}\label{introduction}}

An issue of substantial interest in time series analysis is, whether
there exists any meaningful equilibrium relationship between two or more
time series variables. Various hypothesis tests have been suggested for
testing this so-called cointegration relationship, with the null
hypothesis of no cointegration. Their local power, however, relies
mostly on a specific nuisance parameter, namely the squared long-run
correlations of error terms driving the variables. This may lead to
inconclusive results, as one test may reject the null hypothesis, while
others accept. The detection of cointegration relationships among time
series variables is therefore complicated. Furthermore, the decision for
an applicable test poses a challenge for the practitioner.

An approach for resolving this issue might be a combination of the
different tests. Bayer and Hanck suggest a method for providing meta
tests, with high power for all forms of the nuisance parameter
\autocite{Bayerhanck2009}. Their approach is based on Fisher's
Chi-squared test \autocite{Fisher1925}. It can be shown, that this
provides an unambiguous test decision.

So far, there exists a Stata module for computing the above-mentioned
non-cointegration test. However, there is no implementation in R yet.
Therefore, the objective of this work was the development of the R
package \textbf{bayerhanck}, to implement the eponymous test. In section
2, the theoretical background of the combined non-cointegration test
will be explained further. Next, the structure of the associated R
package and its functions will be illustrated. Finally, the performance
of this approach will be evaluated.

\hypertarget{theory-of-non-cointegration-tests}{%
\section{Theory of Non-Cointegration
Tests}\label{theory-of-non-cointegration-tests}}

As mentioned above, \textcite{Bayerhanck2009} point out, that there is
no uniformly most powerful test for cointegration. This is due to the
fact, that all common tests are affected by the nuisance parameter
\(R^2\), where \(R^2\) is defined as the squared correlation of
\(\pmb{\nu}_{1t}\) with \(\nu_{2t}\).\footnote{\(R^2:= \omega_{12}^{'} \pmb{\Omega}_{11}^{-1}\omega_{12} / \omega_{22}\)}
Furthermore, \textcite{pesavento2004} has shown that for varying \(R^2\)
different tests are more powerful. Hence it is not clear which test
should be applied to the time series data.

To resolve these issues, Bayer and Hanck propose a combination of
various cointegration tests. To derive the properties of the underlying
tests and their combination, Bayer and Hanck use the following model by
\textcite{pesavento2004}:

\begin{align}
\label{eq:regressor}
\triangle \pmb{x}_t = \pmb{\tau}_1 &+  \pmb{\nu}_{1t}\\
\label{eq:y}
y_t  = \left(\mu_2 - \pmb{\theta}' \pmb{\mu}_1 \right) &+ \left(\tau_2 - \pmb{\theta}' \pmb{\tau}_1 \right)t + \pmb{\theta}' x_t + u_t \nonumber
\end{align}

where \(u_t = \rho u_{t-1} + v_{2t}\).

Equation \eqref{eq:regressor} describes the regressors, whereas equation
\eqref{eq:y} describes the possible cointgeration vector. The observed
sample can be denoted as \(\pmb{z}_0, \ldots , \pmb{z}_T\), where
\(\pmb{z}_t = (\pmb{x}_t^{'}, y_t)\).

Under the following two assumptions and \(\rho = 1\) the
\(\mathcal{H}_0\) hypothesis holds for the vector \(\pmb{z}_t\):

\begin{itemize}
  \item[] Assumption 1: 
  \begin{itemize}
    \item[] $\{ \pmb{\nu}_t \}$ satisfied a Functional CLT, i.e. $\displaystyle T^{-1/2} \sum_{t = 1}^{[\cdot T]} \pmb{\nu}_t \Rightarrow \pmb{\Omega}^{1/2} \pmb{W}(\cdot)$, i.e $\pmb{z}_t$ is $I(1)$  
  \end{itemize}
  \item[] Assumption 2:
  \begin{itemize}
    \item[] There are no cointegrated relationships among the variables in $\pmb{x}_t$
  \end{itemize}
\end{itemize}

Two essential properties of this model are: firstly, the local power of
the underlying tests only depends on the parameters \(c := T(\rho -1)\)
and \(R^2\). Secondly, even asymptotically these statistics are only
weakly correlated under \(\mathcal{H}_0\). \autocite{gregory_mixed_2004}

With these central findings it may be possible to achieve a more robust
test by combining individual underlying tests. Bayer and Hanck utilise
the cointegration tests of Engel and Granger
\autocite{Englegranger1987}, Johansen \autocite{Johansen1988}, Boswijk
\autocite{Boswijk1994} and Banerjee \autocite{Banerjee1998} as their
underlying tests.

Each of these tests has a slightly different approach to testing for a
possible cointegration relationship. The Engle-Granger Test tests the
null of non cointegration against the alternative hypothesis of at least
one cointegration relationship. For this, a simple two-step procedure is
used. Firstly, the residuals \(\hat{u}_t\) of a regression of the
dependent variable on the independent variable is computed. Then, the
first differences of these residuals are regressed on the lagged
residuals. The test statistic for the Engle-Granger Test is then the
\(t\)-statistic \(t^{ADF}_\gamma\) on \(\gamma\) in the aforesaid second
regression
\(\triangle \hat{u}_t = \gamma \hat{u}_{t-1} + \sum_{p = 1}^{P-1}\nu_p \triangle \hat{u}_{t-p} +\epsilon_t\).
\autocite{Englegranger1987}

As opposed to the Engle-Granger Test, \textcite{Johansen1988}'s
cointegration test is able to test for \(h\) multiple cointegration
relationships. The null hypothesis is \(h = 0\). To implement this test,
a vector error correction model

\begin{align}
\triangle {\bf z}_t = {\bf \Pi z}_{t-1} + \sum^{P-1}_{p=1}\pmb{\Gamma}_p \triangle{\bf z}_{t-p} + {\bf d}_t + \pmb{\epsilon}_t
\end{align}

needs to estimated first. Bayer and Hanck employ the test statstic
\(\lambda_{\max} (h) = - T \ ln(1 - \hat{\pi}_1)\), where
\(\hat{\pi}_1\) denotes the largest solution of
\(|\pi \pmb{S}_{11} - \pmb{S}_{10} \pmb{S}_{00}^{-1} \pmb{S}_{01}|= 0\).\footnote{\(\pmb{S}_{ij}\)
  denotes the moment matrices of redcued rank regression residuals.}

The cointegration tests of \textcite{Boswijk1994} and
\textcite{Banerjee1998} are based on error correction model tests.
Boswijk's test statistic \(\hat{F}\) is the Wald statistic for
\(\mathcal{H}_0: (\varphi_0, \pmb{\phi}_1^{'}) = 0\), whereas Banerjee
\emph{et al.}'s test statistic is the \(t_{\gamma}^{ECR}\) ratio for
\(\mathcal{H}_0 : \varphi_0 = 0\) of the error correction model. To
derive the error correction model, the usual least squares estimate is
used:
\(\triangle y_t = d_t + \pi_{0x}^{'} \triangle \pmb{x}_t + \varphi_0 y_{t-1} + \pmb{\varphi}_{1}^{'} \pmb{x}_{t-1} + \sum_{P = 1}^{P} \left( \pmb{\pi}_{px}^{'} \triangle \pmb{x}_{t-p} + \pi_{py} \triangle y_{t - p} \right) + \epsilon_t\).

From now on, \(t_i\) denotes the test statistic of the underlying test
\(i\). If the this test rejects for large (small) values, take
\(\xi_i := t_i (-\xi_i = t_i)\). To combine these tests into a more
powerful joint non cointegration test, Bayer and Hanck use an aggregator
employed by \textcite{Fisher1925}'s Chi-squared test. The test statistic
\(\tilde{\chi}_{\mathcal{I}}^2\) is than defined as:

\begin{align}
  \label{eq:bayer-hanck}
  \tilde{\chi}_{\mathcal{I}}^{2} := -2 \sum_{i \in \mathcal{I}} \ln(p_i),
\end{align}

where \(\mathcal{I}\) is the index set of the \(\xi_i\).

The test statistic has the property, that the distribution under the
\(\mathcal{H}_0\) hypothesis converges to a random variable
\(\mathcal{F}_{\mathcal{I}}\)
\(\left(\tilde{\chi}_{\mathcal{I}}^{2} \rightarrow_{d} \mathcal{F}_{\mathcal{I}} \right)\).
This guarantees, that the statistic has a well-defined asymptotic null
distribution. According to \textcite{Bayerhanck2009}, this null
distribution is nuisance parameter free and only depends on how many
tests are combined and on the \(\xi_i\). Consequently, the joint
\(\mathcal{H_0}\) hypotheses can be simulated.

Under the alternative hypotheses \(\mathcal{H}_1\) the test statistic
probability converges to
\(\tilde{\chi}_{\mathcal{I}}^{2} \rightarrow_P \infty\), which means
that the test statistic is consistent if at least one of combined tests
is consistend.

\hypertarget{implementation-of-the-package-bayerhanck-in-r}{%
\section{\texorpdfstring{Implementation of the Package
\textbf{bayerhanck} in
R}{Implementation of the Package bayerhanck in R}}\label{implementation-of-the-package-bayerhanck-in-r}}

The package consists of four functions for the underlying tests, as well
as the function for the combined test. Furthermore, the cumulative
distribution function of the null hypothesis can be plotted. The package
features will be illustrated by using real data from the lutkepohl
dataset.

\hypertarget{implementation-of-the-underlying-tests}{%
\subsection{Implementation of the underlying
Tests}\label{implementation-of-the-underlying-tests}}

The underlying tests can be carried out by their eponymous commands,
namely \texttt{englegranger()}, \texttt{johansen()}, \texttt{banerjee()}
and \texttt{boswijk()}. The former two partly rely on already
implemented functions from the packages \textbf{urca} and
\textbf{tsDyn}. Due to the absence of associated functions for the
latter two, those had to be programmed manually. All functions take
several arguments, from which only \texttt{formula} and \texttt{data}
need to be filled out by the practitioner. Further arguments orientate
themselves on the default values defined in the Stata implementation.
For the argument \texttt{lags}, which determines the number of lags to
be included in the model, the default value is therefore set to 1. The
argument \texttt{trend} describes the deterministic components of the
model. The practitioner may choose from \texttt{none}, for no
deterministics, \texttt{const}, for a (unrestricted) constant and
\texttt{trend}, for a (unrestricted) constant plus (unrestricted) trend.
The default value is set to \texttt{const}. The functions of the
underlying tests all return an object of classes \texttt{co.test} and
\texttt{list}. The console also returns the value of the test statistic,
as well as the name of the test executed.

In accordance with the previously explained structure, the function
\texttt{englegranger()} takes the form:

\begin{Shaded}
\begin{Highlighting}[]
\KeywordTok{englegranger}\NormalTok{(formula, data, }\DataTypeTok{lags =} \DecValTok{1}\NormalTok{, }\DataTypeTok{trend =} \StringTok{"const"}\NormalTok{)}
\end{Highlighting}
\end{Shaded}

The structure of this function is orientated towards the implementation
of the Engle-Granger test in the aforesaid Stata module. Therefore, none
of the various existing functions in R is used. Firstly, a linear
regression is performed, according to the formula entered in
\texttt{englegranger()}. Next, an augmented Dickey-Fuller test is
applied to the residuals from this regression. For this, the function
\texttt{ur.df()} from the \textbf{urca} package is used. The output
value of the test statistic is then used as the test statistic of the
Engle-Granger test.

The function for the Johansen test contains the additional argument
\texttt{type}, which specifies if an maximum eigenvalue or a trace test
should be conducted. Therefore, the options are either \texttt{trace} or
\texttt{eigen}, with \texttt{eigen} being the default choice. The
structure of the function takes the form:

\begin{Shaded}
\begin{Highlighting}[]
\KeywordTok{johansen}\NormalTok{(formula, data, }\DataTypeTok{type =} \StringTok{"eigen"}\NormalTok{, }\DataTypeTok{lags =} \DecValTok{1}\NormalTok{, }\DataTypeTok{trend =} \StringTok{"const"}\NormalTok{)}
\end{Highlighting}
\end{Shaded}

Firstly, the function estimates a VECM by Johansen (MLE) method with
\texttt{VECM} from the package \textbf{tsDyn}. Then, a test of the
cointegrating rank is conducted. For this, the function
\texttt{rank.test()} from the same package is used. In comparison with
similar functions \texttt{rank.test()} possesses the advantage of the
possibility to select the unrestricted constant or trend. Here, the
maximum eigenvalue is used as the output test statistic of
\texttt{johansen()}.

The construction of the functions for the Banerjee and Boswijk test was
almost identical. Therefore, both functions will be illustrated in a
single step. The arguments of both functions are identical to those from
\texttt{englegranger()}:

\begin{Shaded}
\begin{Highlighting}[]
\KeywordTok{banerjee}\NormalTok{(formula, data, }\DataTypeTok{lags =} \DecValTok{1}\NormalTok{, }\DataTypeTok{trend =} \StringTok{"const"}\NormalTok{)}
\KeywordTok{boswijk}\NormalTok{(formula, data, }\DataTypeTok{lags =} \DecValTok{1}\NormalTok{, }\DataTypeTok{trend =} \StringTok{"const"}\NormalTok{)}
\end{Highlighting}
\end{Shaded}

For the construction of those functions, first differences had to be
taken of the dependent, as well as the independent variables.
Furthermore, a matrix of the lagged values of all variables in first
differences was constructed.

\textcolor{red}{Weitere Beschreibung Banerjee/Boswijk (Jens)}

From this linear Regression the coefficients, as well as the
covariance-matrix is extracted. Now, the test statistic of both tests
can be calculated. For the test statistic of the Banerjee test, the
coefficient of \textcolor{red}{Position} has to be divided by the
associated entry in the covariance-matrix. For the test statistic of the
Boswijk test, all coefficients will firstly be matrix multiplied with
the inverse covariance-matrix. Next, this product will then again be
matrix multiplied with the coefficients.

\hypertarget{implementation-of-the-function-bayerhanck}{%
\subsection{\texorpdfstring{Implementation of the function
\texttt{bayerhanck()}}{Implementation of the function bayerhanck()}}\label{implementation-of-the-function-bayerhanck}}

\begin{Shaded}
\begin{Highlighting}[]
\KeywordTok{bayerhanck}\NormalTok{(formula, data, }\DataTypeTok{lags =} \DecValTok{1}\NormalTok{, }\DataTypeTok{trend =} \StringTok{"const"}\NormalTok{, }
           \DataTypeTok{test =} \StringTok{"all"}\NormalTok{, }\DataTypeTok{crit =} \FloatTok{0.05}\NormalTok{)}
\end{Highlighting}
\end{Shaded}

The combined non-cointegration test can be carried out by the command
\texttt{bayerhanck()} which accepts all the arguments of the underlying
tests. Also the additional arguments \texttt{test} and \texttt{crit} can
be passt to the function call. The argument \texttt{test} etermines
which set of underlying tests should be executed. Here users can choose
between the default \texttt{all} (performs all the above tests), or
\texttt{eg-j}, which only carries out the Engel-Granger and Johansen
test. For the argument \texttt{crit}, which sets the significance level
of the test, the user may choose between \texttt{0.10}, the default
\texttt{0.05}, or \texttt{0.01}, corresponding with 10 \%, 5 \%, or 1\%
significance.

After the selected underlying tests have been performed, the test
statistics of each test are obtained and the p-values are calculated by
using the corresponding empirical null distribution which depends on the
number of variables in the VAR and the trend specification.

In the following step, the individual p-values are logarithmized ,
summed and then taken with minus 2 times, in accorance with equation
\eqref{eq:bayer-hanck}, to obtain the test statistics. In the last step,
the critical value is selected based on the specified significance level
from the empirical null distribution of the combined test, which depends
on the number of regressors and the trend type.

\hypertarget{conclusion}{%
\section{Conclusion}\label{conclusion}}

\newpage
\renewcommand*{\mkbibnamefamily}[1]{\textbf{#1}}
\renewcommand*{\mkbibnamegiven}[1]{\textbf{#1}}
\renewcommand*{\mkbibnameprefix}[1]{\textbf{#1}}
\renewcommand*{\mkbibnamesuffix}[1]{\textbf{#1}}
\printbibliography[title=References]

\newpage
\textbf{Eidesstattliche Versicherung}

\bigskip

Ich versichere an Eides statt durch meine Unterschrift, dass ich die vorstehende Arbeit selbständig und ohne fremde Hilfe angefertigt und alle Stellen, die ich wörtlich oder annähernd wörtlich aus Veröffentlichungen entnommen habe, als solche kenntlich gemacht habe, mich auch keiner anderen als der angegebenen Literatur oder sonstiger Hilfsmittel bedient habe. Die Arbeit hat in dieser oder ähnlicher Form noch keiner anderen Prüfungsbehörde vorgelegen.

\vspace{1cm}
\rule{0pt}{2\baselineskip} %
\par\noindent\makebox[2.25in]{\indent Essen, den \hrulefill} \hfill\makebox[2.25in]{\hrulefill}%
\par\noindent\makebox[2.25in][l]{} \hfill\makebox[2.25in][c]{Name}%


\end{document}
