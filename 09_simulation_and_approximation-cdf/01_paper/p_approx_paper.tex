\documentclass[12pt,a4paper]{article}
\usepackage{lmodern}

\usepackage{placeins}
\usepackage{amssymb,amsmath}
\usepackage{ifxetex,ifluatex}
\usepackage{fixltx2e} % provides \textsubscript
\ifnum 0\ifxetex 1\fi\ifluatex 1\fi=0 % if pdftex
  \usepackage[T1]{fontenc}
  \usepackage[utf8]{inputenc}
\else % if luatex or xelatex
  \ifxetex
    \usepackage{mathspec}
    \usepackage{xltxtra,xunicode}
  \else
    \usepackage{fontspec}
  \fi
  \defaultfontfeatures{Mapping=tex-text,Scale=MatchLowercase}
  \newcommand{\euro}{€}
\fi
% use upquote if available, for straight quotes in verbatim environments
\IfFileExists{upquote.sty}{\usepackage{upquote}}{}
% use microtype if available
\IfFileExists{microtype.sty}{%
\usepackage{microtype}
\UseMicrotypeSet[protrusion]{basicmath} % disable protrusion for tt fonts
}{}
\usepackage[lmargin = 5cm,rmargin = 2.5cm,tmargin = 2.5cm,bmargin = 2.5cm]{geometry}

% Figure Placement:
\usepackage{float}
\let\origfigure\figure
\let\endorigfigure\endfigure
\renewenvironment{figure}[1][2] {
    \expandafter\origfigure\expandafter[H]
} {
    \endorigfigure
}

%%%% Jens %%%%
\DeclareMathOperator*{\argmax}{arg\,max}
\DeclareMathOperator*{\argmin}{arg\,min}

\usepackage{numprint}
\npthousandsep{\,}

%% citation setup
\usepackage{csquotes}

\usepackage[backend=biber, maxbibnames = 99, style = apa]{biblatex}
\setlength\bibitemsep{1.5\itemsep}
\addbibresource{R_packages.bib}
\bibliography{references.bib}
\usepackage{graphicx}
\makeatletter
\def\maxwidth{\ifdim\Gin@nat@width>\linewidth\linewidth\else\Gin@nat@width\fi}
\def\maxheight{\ifdim\Gin@nat@height>\textheight\textheight\else\Gin@nat@height\fi}
\makeatother
% Scale images if necessary, so that they will not overflow the page
% margins by default, and it is still possible to overwrite the defaults
% using explicit options in \includegraphics[width, height, ...]{}
\setkeys{Gin}{width=\maxwidth,height=\maxheight,keepaspectratio}
\ifxetex
  \usepackage[setpagesize=false, % page size defined by xetex
              unicode=false, % unicode breaks when used with xetex
              xetex]{hyperref}
\else
  \usepackage[unicode=true, linktocpage = TRUE]{hyperref}
\fi
\hypersetup{breaklinks=true,
            bookmarks=true,
            pdfauthor={Jens Klenke and Janine Langerbein},
            pdftitle={P-Approximation},
            colorlinks=true,
            citecolor=black,
            urlcolor=black,
            linkcolor=black,
            pdfborder={0 0 0}}
\urlstyle{same}  % don't use monospace font for urls
\setlength{\parindent}{0pt}
\setlength{\parskip}{6pt plus 2pt minus 1pt}
\setlength{\emergencystretch}{3em}  % prevent overfull lines
\setcounter{secnumdepth}{5}

%%% Use protect on footnotes to avoid problems with footnotes in titles
\let\rmarkdownfootnote\footnote%
\def\footnote{\protect\rmarkdownfootnote}

%%% Change title format to be more compact
\usepackage{titling}

% Create subtitle command for use in maketitle
\newcommand{\subtitle}[1]{
  \posttitle{
    \begin{center}\large#1\end{center}
    }
}

\setlength{\droptitle}{-2em}
  \title{P-Approximation}
  \pretitle{\vspace{\droptitle}\centering\huge}
  \posttitle{\par}
\subtitle{Seminar in Econometrics}
  \author{Jens Klenke and Janine Langerbein}
  \preauthor{\centering\large\emph}
  \postauthor{\par}
  \predate{\centering\large\emph}
  \postdate{\par}
  \date{today}

\usepackage{booktabs}
\usepackage{longtable}
\usepackage{array}
\usepackage{multirow}
\usepackage{wrapfig}
\usepackage{float}
\usepackage{colortbl}
\usepackage{pdflscape}
\usepackage{tabu}
\usepackage{threeparttable}
\usepackage{threeparttablex}
\usepackage[normalem]{ulem}
\usepackage{makecell}
\usepackage{xcolor}

%% linespread settings

\usepackage{setspace}

\onehalfspacing

% Language Setup

\usepackage{ifthen}
\usepackage{iflang}
\usepackage[super]{nth}
\usepackage[ngerman, english]{babel}

%Acronyms
\usepackage[printonlyused, withpage, nohyperlinks]{acronym}
\usepackage{changepage}

% Multicols for the Title page
\usepackage{multicol}

\begin{document}

\selectlanguage{english}

%%%%%%%%%%%%%% Jens %%%%%
\numberwithin{equation}{section}


%\maketitle

\begin{titlepage}
  \noindent\begin{minipage}{0.6\textwidth}
	  \IfLanguageName{english}{University of Duisburg-Essen}{Universität Duisburg-Essen}\\
	  \IfLanguageName{english}{Faculty of Business Administration and Economics}{Fakultät für Wirtschaftswissensschaften}\\
	  \IfLanguageName{english}{Chair of Econometrics}{Lehrstuhl für Ökonometrie}\\
  \end{minipage}
	\begin{minipage}{0.4\textwidth}
	  \begin{flushright}
  	  \vspace{-0.5cm}
      \IfLanguageName{english}{\includegraphics*[width=5cm]{Includes/duelogo_en.png}}{\includegraphics*[width=5cm]{Includes/duelogo_de.png}}
	  \end{flushright}
	\end{minipage}
  \\
  \vspace{0.25cm}
  \begin{center}
  \huge{P-Approximation}\\
  \vspace{.25cm}
  \Large{Seminar in Econometrics}\\
  \vspace{0.5cm}
  \large{Term Paper}\\
  \vspace{0.5cm}
  \large{  \IfLanguageName{english}{Submitted to the Faculty of \\  Business Administration and Economics  \\at the \\University of Duisburg-Essen}{Vorgelegt der \\Fakultät für Wirtschaftswissenschaften der \\ Universität Duisburg-Essen}\\}
  \vspace{0.75cm}
  \large{\IfLanguageName{english}{from:}{von:}}\\
  \vspace{0.5cm}
  Jens Klenke and Janine Langerbein\\
  \end{center}
  %\vspace{2cm}
  \vfill
  \hrulefill

  \noindent\begin{minipage}[t]{0.3\textwidth}
  \IfLanguageName{english}{Reviewer:}{Erstgutachter:}
  \end{minipage}
  \begin{minipage}[t]{0.7\textwidth}
  \hspace{1cm}Christoph Hanck
  \end{minipage}

  \noindent\begin{minipage}[t]{0.3\textwidth}
  \IfLanguageName{english}{Deadline:}{Abgabefrist:}
  \end{minipage}
  \begin{minipage}[t]{0.7\textwidth}
  \hspace{1cm}Jan.~17th 2020
  \end{minipage}

  \hrulefill

  \begin{multicols}{3}
  
  \begin{scriptsize}
  
  Name:

  Matriculation Number:

  E-Mail:

  Study Path:

  Semester:

  Graduation (est.):
 
  \columnbreak

  Jens Klenke

  3071594
  
  jens.klenke@stud.uni-due.de

  M.Sc. Economics

  \nth{5}

  Winter Term 2020
  
  \columnbreak
  
  Janine Langerbein

  307
  
  janine.langerbein@stud.uni-due.de

  M.Sc. Economics

  \nth{5}

  Winter Term 2020
  
  \end{scriptsize}
  
  \end{multicols}
  
  

\end{titlepage}

\newgeometry{top=2cm, left = 5cm, right = 2.5cm, bottom = 2.5cm}


\pagenumbering{Roman}
{
\hypersetup{linkcolor=black}

\setcounter{tocdepth}{3}
\tableofcontents
}

\newpage
\listoffigures
\addcontentsline{toc}{section}{List of Figures}

%\newpage
\listoftables
\addcontentsline{toc}{section}{List of Tables}

\section*{List of Abbreviations}
\addcontentsline{toc}{section}{List of Abbreviations}

\begin{adjustwidth}{1.5em}{0pt}

\begin{acronym}[dummyyyy]
 \acro{bagging}{Bootstrap Aggregation}
 \acro{LASSO}{Least Absolute Shrinkage and Selection Operator}
 \acro{OLS}{ordinary least squares}
 \acro{pcr}{Principal Components Regression}
 \acro{pls}{Partial Least Squares}
 \acro{RMSE}{Root Mean Squared Error}
 \acro{MCMC}{Markov chain Monte Carlo} 
 \acro{i.i.d.}{independent and identically distributed}
 \acroplural{LRG}[LRG]{längefristige Refinanzierungsgeschäfte}

%Falls eine Abkürzung in der Mehrzahl nicht einfach auf "s" endet muss das speziell eingestellt werden.
% \acro{slmtA}{super lange mega tolle Abkürzung} %Einzahl
 %\acroplural{slmtA}[slmtAs]{super lange mega tolle Abkürzungen} %Mehrzahl
 \acro{dummyyyy}{dummyyy}
\end{acronym}

\end{adjustwidth}

\restoregeometry

\newpage
\pagenumbering{arabic} %Roman arabic

\hypertarget{introduction}{%
\section{Introduction}\label{introduction}}

Meta tests have been shown to be a powerful tool when testing for the
null of non-cointegration. The distribution of their test statistic,
however, is mostly not available in closed form. This might pose
difficulties when implementing the meta tests in econometric software
packages, as one has to include tables of critical values and p-values
for each combination of the underlying tests. Software package size
limitations are therefore quickly exceeded.

In this paper we propose supervised Machine Learning Algorithms to
approximate the p-values of the meta test by Bayer and Hanck
\autocite*{Bayerhanck_2012} which tests for the null of
non-cointegration. This approach might reduce the size of associated
software packages considerably. The algorithms are trained on simulated
data for various specifications of the aforementioned test.

\textcolor{red}{Ergebnis der Models (1-2 Sätze)}

\textcolor{red}{Inhalt Paper}

\hypertarget{bayer-hanck-test}{%
\section{Bayer Hanck Test}\label{bayer-hanck-test}}

The choice as to which of the available cointegration tests to use is a
recurrent issue in econometric time series analysis.
\textcite{Bayerhanck_2012} propose powerful meta tests which provide
unambiguous test decisions. They combine several residual- and
system-based tests in the manner of Fisher's \autocite*{Fisher_1932}
Chi-squared test.

Bayer and Hanck build their work on results from
\textcite{Pesavento_2004}, who defines the underlying model as
\(z'_t = [x'_t, y_t]\). \(x_t\), an \(n_1 \times 1\) vector, describes
the regressor dynamics, while \(y_t\) is a scalar which defines the
cointegrating relation. They can be written as

\begin{align}
\Delta x_t &= \tau_1 + v_{1t}, \\
y_t &= (\mu_2 - \gamma' \mu_1) + (\tau_2 - \gamma' \tau_1) t + \gamma' x_t + u_t, \\
u_t &= \rho u_{t-1} + v_{2t}.
\end{align}

\(\mu_1\), \(\mu_2\) \(\tau_1\) and \(\tau_2\) are the deterministic
parts of the model. They are subject to the following restrictions: (i)
\(\mu_2 - \gamma' \mu_1\) and \(\tau = 0\) which translates to no
deterministics, (ii) \(\tau = 0\) which corresponds to a constant in the
cointegrating vector, (iii) \(\tau_2 - \gamma' \tau_1 = 0\), a constant
plus trend.

\(v_t = [v'_{1t} v_{2t}]'\) with \(\Omega\) the long-run covariance
matrix of \(v_t\). For derivation of \(v_t\) see
\textcite{Pesavento_2004}. Pesavento shows that \{\(v_t\)\} satisfies an
FCLT,
i.e.~\(T^{-1/2} \sum^{[T \cdot]}_{t=1} v_t \Rightarrow \Omega^{1/2} W(\cdot)\).
It is further assumed that the \(x_t\) are not cointegrated.

It clearly follows from (2.3) that \(z_t\) is cointegrated if
\(\rho < 1\). Hence the null hypothesis of no cointegration is
\(H_0: p = 1\).

Furthermore, Pesavento introduces two other parameters. First,
\(\text{R}^2\) measures the squared correlation of \(v_{1t}\) and
\(v_{2t}\). It can be interpreted as the influence of the right-hand
side variables in (2.2). It ranks between zero and one. When there is no
long-run correlation between those variables and the errors from the
cointegration regression, \(\text{R}^2\) equals zero. Secondly, the
number of lags is approximated by a finite number \(k\).

\textcolor{red}{Assumptions (BH S. 84)?}

Bayer and Hanck's \autocite*{Bayerhanck_2012} meta test combines the
test statistics of four stand-alone tests. Namely, these are the tests
of \textcite{Englegranger_1987}, \textcite{Johansen_1988},
\textcite{Boswijk_1994} and \textcite{Banerjee_1998}.

\textcolor{red}{Engle-Granger}

\textcolor{red}{Johansen}

\textcolor{red}{Banerjee and Boswijk}

To combine the results from the underlying tests Bayer and Hanck draw
upon Fisher's combined probability test \autocite{Fisher_1932}. It
merges the tests using the formula

\begin{equation}
\tilde{\chi}^2_{\mathcal{I}} := -2 \sum_{i \in \mathcal{I}} \ln{(p_i)}. 
\label{Fisher}
\end{equation}

Let \(t_i\) be the \(i^{th}\) test statistic. If test \(i\) rejects for
large values, take \(\xi_i := t_i\). If test \(i\) rejects for small
values, take \(-\xi_i := t_i\). With
\(\Xi_i(x) := \text{Pr}_{\mathcal{H_0}}(\xi_i \geq x)\) the p-value of
the \(i^{th}\) test is \(p_i := \Xi_i(\xi_i)\).

\textcite{Fisher_1932} shows that under the assumption of independence
the null distribution of \(\tilde{\chi}^2_{\mathcal{I}}\) follows a
chi-squared distribution with \(2\mathcal{I}\) degrees of freedom. If
this assumption is violated the null distribution is less evident. Here,
the latter case occurs, as the \(\xi_i\) are not independent. The
\(\tilde{\chi}^2_{\mathcal{I}}\), however, have well-defined asymptotic
null distributions \(F_{\mathcal{F_I}}\), as
\(\tilde{\chi}^2_{\mathcal{I}} \rightarrow_d \mathcal{F_I}\) under
\(\mathcal{H}_0\) if \(T \rightarrow \infty\), with \(\mathcal{F_I}\)
some random variable. It is therefore feasible to simulate the joint
null distribution of the \(\xi_i\) to obtain the distribution
\(F_{\mathcal{F_I}}\) of \ref{Fisher}. The \(F_{\mathcal{F_I}}\) depend
on which and how many tests are combined. The distributions of the
\(\xi_i\) depend on \(K-1\) and the deterministic case.

\hypertarget{simulation}{%
\section{Simulation}\label{simulation}}

\hypertarget{models}{%
\section{Models}\label{models}}

\hypertarget{package}{%
\section{Package}\label{package}}

\pagebreak

\pagenumbering{Roman}

\addcontentsline{toc}{section}{References}
\printbibliography[title = References]
\cleardoublepage

\begin{refsection}
\nocite{R-base}
\nocite{R-stargazer}
\nocite{R-stringr}
\nocite{R-tidyr}
\nocite{R-dplyr}
\nocite{R-glmnet}
\nocite{R-class}
\nocite{R-MASS}
\nocite{R-plm}
\nocite{R-leaps}
\nocite{R-caret}
\nocite{R-tree}
\nocite{R-gbm}
\nocite{R-plotmo}
\nocite{R-pls}
\nocite{R-splines}
\nocite{R-tictoc}
\nocite{R-plotly}
\nocite{R-inspectdf}
\nocite{R-rpart}
\nocite{R-rpart.plot}
\nocite{R-stargazer}
\nocite{R-knitr}
\nocite{R-purrr}
\nocite{R-randomForest}
\nocite{R-rstudioapi}





\nocite{R-Studio}

\printbibliography[title = Software-References]
\addcontentsline{toc}{section}{Software-References}
\end{refsection}

\cleardoublepage
\appendix
\setcounter{table}{0}
\setcounter{figure}{0}
\renewcommand{\thetable}{A\arabic{table}}
\renewcommand{\thefigure}{A\arabic{figure}}

\newgeometry{top = 2.5cm, left = 5cm, right = 2.5cm, bottom = 2cm}

\hypertarget{appendices}{%
\section{Appendices}\label{appendices}}

Better sorting of the Appendix

\restoregeometry

\cleardoublepage
\newpage
\renewcommand*{\mkbibnamefamily}[1]{\textbf{#1}}
\renewcommand*{\mkbibnamegiven}[1]{\textbf{#1}}
\renewcommand*{\mkbibnameprefix}[1]{\textbf{#1}}
\renewcommand*{\mkbibnamesuffix}[1]{\textbf{#1}}


% \printbibliography[title=References]
%\pagenumbering{arabic}


\newpage
\textbf{Eidesstattliche Versicherung}

\bigskip

Ich versichere an Eides statt durch meine Unterschrift, dass ich die vorstehende Arbeit selbständig und ohne fremde Hilfe angefertigt und alle Stellen, die ich wörtlich oder annähernd wörtlich aus Veröffentlichungen entnommen habe, als solche kenntlich gemacht habe, mich auch keiner anderen als der angegebenen Literatur oder sonstiger Hilfsmittel bedient habe. Die Arbeit hat in dieser oder ähnlicher Form noch keiner anderen Prüfungsbehörde vorgelegen.

\vspace{1cm}
\rule{0pt}{2\baselineskip} %
\par\noindent\makebox[2.25in]{\indent Essen, den \hrulefill} \hfill\makebox[2.25in]{\hrulefill}%
\par\noindent\makebox[2.25in][l]{} \hfill\makebox[2.25in][c]{Jens Klenke and Janine Langerbein}%


\end{document}
