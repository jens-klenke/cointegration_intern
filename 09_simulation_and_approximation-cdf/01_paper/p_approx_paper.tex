\documentclass[12pt,a4paper]{article}
\usepackage{lmodern}

\usepackage{placeins}
\usepackage{amssymb,amsmath}
\usepackage{ifxetex,ifluatex}
\usepackage{fixltx2e} % provides \textsubscript
\ifnum 0\ifxetex 1\fi\ifluatex 1\fi=0 % if pdftex
  \usepackage[T1]{fontenc}
  \usepackage[utf8]{inputenc}
\else % if luatex or xelatex
  \ifxetex
    \usepackage{mathspec}
    \usepackage{xltxtra,xunicode}
  \else
    \usepackage{fontspec}
  \fi
  \defaultfontfeatures{Mapping=tex-text,Scale=MatchLowercase}
  \newcommand{\euro}{€}
\fi
% use upquote if available, for straight quotes in verbatim environments
\IfFileExists{upquote.sty}{\usepackage{upquote}}{}
% use microtype if available
\IfFileExists{microtype.sty}{%
\usepackage{microtype}
\UseMicrotypeSet[protrusion]{basicmath} % disable protrusion for tt fonts
}{}
\usepackage[lmargin = 5cm,rmargin = 2.5cm,tmargin = 2.5cm,bmargin = 2.5cm]{geometry}

% Figure Placement:
\usepackage{float}
\let\origfigure\figure
\let\endorigfigure\endfigure
\renewenvironment{figure}[1][2] {
    \expandafter\origfigure\expandafter[H]
} {
    \endorigfigure
}

%%%% Jens %%%%
\DeclareMathOperator*{\argmax}{arg\,max}
\DeclareMathOperator*{\argmin}{arg\,min}
\DeclareMathOperator{\bc}{bc}
\DeclareMathOperator{\poly}{poly}

%%%% table %%%%
\usepackage{lscape}
\renewcommand{\arraystretch}{1.5}

\usepackage{numprint}
\npthousandsep{\,}

%% citation setup
\usepackage{csquotes}

\usepackage[backend=biber, maxbibnames = 99, style = apa]{biblatex}
\setlength\bibitemsep{1.5\itemsep}
\addbibresource{R_packages.bib}
\bibliography{references.bib}
\usepackage{graphicx}
\makeatletter
\def\maxwidth{\ifdim\Gin@nat@width>\linewidth\linewidth\else\Gin@nat@width\fi}
\def\maxheight{\ifdim\Gin@nat@height>\textheight\textheight\else\Gin@nat@height\fi}
\makeatother
% Scale images if necessary, so that they will not overflow the page
% margins by default, and it is still possible to overwrite the defaults
% using explicit options in \includegraphics[width, height, ...]{}
\setkeys{Gin}{width=\maxwidth,height=\maxheight,keepaspectratio}
\ifxetex
  \usepackage[setpagesize=false, % page size defined by xetex
              unicode=false, % unicode breaks when used with xetex
              xetex]{hyperref}
\else
  \usepackage[unicode=true, linktocpage = TRUE]{hyperref}
\fi
\hypersetup{breaklinks=true,
            bookmarks=true,
            pdfauthor={Jens Klenke and Janine Langerbein},
            pdftitle={P-Approximation},
            colorlinks=true,
            citecolor=black,
            urlcolor=black,
            linkcolor=black,
            pdfborder={0 0 0}}
\urlstyle{same}  % don't use monospace font for urls
\setlength{\parindent}{0pt}
\setlength{\parskip}{6pt plus 2pt minus 1pt}
\setlength{\emergencystretch}{3em}  % prevent overfull lines
\setcounter{secnumdepth}{5}

%%% Use protect on footnotes to avoid problems with footnotes in titles
\let\rmarkdownfootnote\footnote%
\def\footnote{\protect\rmarkdownfootnote}

%%% Change title format to be more compact
\usepackage{titling}

% Create subtitle command for use in maketitle
\newcommand{\subtitle}[1]{
  \posttitle{
    \begin{center}\large#1\end{center}
    }
}

\setlength{\droptitle}{-2em}
  \title{P-Approximation}
  \pretitle{\vspace{\droptitle}\centering\huge}
  \posttitle{\par}
\subtitle{Seminar in Econometrics}
  \author{Jens Klenke and Janine Langerbein}
  \preauthor{\centering\large\emph}
  \postauthor{\par}
  \predate{\centering\large\emph}
  \postdate{\par}
  \date{today}

\usepackage{booktabs}
\usepackage{longtable}
\usepackage{array}
\usepackage{multirow}
\usepackage{wrapfig}
\usepackage{float}
\usepackage{colortbl}
\usepackage{pdflscape}
\usepackage{tabu}
\usepackage{threeparttable}
\usepackage{threeparttablex}
\usepackage[normalem]{ulem}
\usepackage{makecell}
\usepackage{xcolor}

%% linespread settings

\usepackage{setspace}

\onehalfspacing

% Language Setup

\usepackage{ifthen}
\usepackage{iflang}
\usepackage[super]{nth}
\usepackage[ngerman, english]{babel}

%Acronyms
\usepackage[printonlyused, withpage, nohyperlinks]{acronym}
\usepackage{changepage}

% Multicols for the Title page
\usepackage{multicol}

\begin{document}

\selectlanguage{english}

%%%%%%%%%%%%%% Jens %%%%%
\numberwithin{equation}{section}


%\maketitle

\begin{titlepage}
  \noindent\begin{minipage}{0.6\textwidth}
	  \IfLanguageName{english}{University of Duisburg-Essen}{Universität Duisburg-Essen}\\
	  \IfLanguageName{english}{Faculty of Business Administration and Economics}{Fakultät für Wirtschaftswissensschaften}\\
	  \IfLanguageName{english}{Chair of Econometrics}{Lehrstuhl für Ökonometrie}\\
  \end{minipage}
	\begin{minipage}{0.4\textwidth}
	  \begin{flushright}
  	  \vspace{-0.5cm}
      \IfLanguageName{english}{\includegraphics*[width=5cm]{Includes/duelogo_en.png}}{\includegraphics*[width=5cm]{Includes/duelogo_de.png}}
	  \end{flushright}
	\end{minipage}
  \\
  \vspace{0.25cm}
  \begin{center}
  \huge{P-Approximation}\\
  \vspace{.25cm}
  \Large{Seminar in Econometrics}\\
  \vspace{0.5cm}
  \large{Term Paper}\\
  \vspace{0.5cm}
  \large{  \IfLanguageName{english}{Submitted to the Faculty of \\  Business Administration and Economics  \\at the \\University of Duisburg-Essen}{Vorgelegt der \\Fakultät für Wirtschaftswissenschaften der \\ Universität Duisburg-Essen}\\}
  \vspace{0.75cm}
  \large{\IfLanguageName{english}{from:}{von:}}\\
  \vspace{0.5cm}
  Jens Klenke and Janine Langerbein\\
  \end{center}
  %\vspace{2cm}
  \vfill
  \hrulefill

  \noindent\begin{minipage}[t]{0.3\textwidth}
  \IfLanguageName{english}{Reviewer:}{Erstgutachter:}
  \end{minipage}
  \begin{minipage}[t]{0.7\textwidth}
  \hspace{1cm}Christoph Hanck
  \end{minipage}

  \noindent\begin{minipage}[t]{0.3\textwidth}
  \IfLanguageName{english}{Deadline:}{Abgabefrist:}
  \end{minipage}
  \begin{minipage}[t]{0.7\textwidth}
  \hspace{1cm}Jan.~17th 2020
  \end{minipage}

  \hrulefill

  \begin{multicols}{3}
  
  \begin{scriptsize}
  
  Name:

  Matriculation Number:

  E-Mail:

  Study Path:

  Semester:

  Graduation (est.):
 
  \columnbreak

  Jens Klenke

  3071594
  
  jens.klenke@stud.uni-due.de

  M.Sc. Economics

  \nth{5}

  Summer Term 2021
  
  \columnbreak
  
  Janine Langerbein

  3061371
  
  janine.langerbein@stud.uni-due.de

  M.Sc. Economics

  \nth{5}

  Summer Term 2021
  
  \end{scriptsize}
  
  \end{multicols}
  
  

\end{titlepage}

\newgeometry{top=2cm, left = 5cm, right = 2.5cm, bottom = 2.5cm}


\pagenumbering{Roman}
{
\hypersetup{linkcolor=black}

\setcounter{tocdepth}{3}
\tableofcontents
}

\newpage
\listoffigures
\addcontentsline{toc}{section}{List of Figures}

%\newpage
\listoftables
\addcontentsline{toc}{section}{List of Tables}

\section*{List of Abbreviations}
\addcontentsline{toc}{section}{List of Abbreviations}

\begin{adjustwidth}{1.5em}{0pt}

\begin{acronym}[dummyyyy]
 \acro{bagging}{Bootstrap Aggregation}
 \acro{FCLT}{functional central limit theorem}
 \acro{LASSO}{Least Absolute Shrinkage and Selection Operator}
 \acro{OLS}{ordinary least squares}
 \acro{pcr}{Principal Components Regression}
 \acro{pls}{Partial Least Squares}
 \acro{RMSE}{Root Mean Squared Error}
 \acro{MCMC}{Markov chain Monte Carlo} 
 \acro{i.i.d.}{independent and identically distributed}
 \acroplural{LRG}[LRG]{längefristige Refinanzierungsgeschäfte}

%Falls eine Abkürzung in der Mehrzahl nicht einfach auf "s" endet muss das speziell eingestellt werden.
% \acro{slmtA}{super lange mega tolle Abkürzung} %Einzahl
 %\acroplural{slmtA}[slmtAs]{super lange mega tolle Abkürzungen} %Mehrzahl
 \acro{dummyyyy}{dummyyy}
\end{acronym}

\end{adjustwidth}

\restoregeometry

\newpage
\pagenumbering{arabic} %Roman arabic

\hypertarget{introduction}{%
\section{Introduction}\label{introduction}}

Meta tests have been shown to be a powerful tool when testing for the
null of non-cointegration. The distribution of their test statistic,
however, is mostly not available in closed form. This might pose
difficulties when implementing the meta tests in econometric software
packages, as one has to include the full null distribution for each
combination of the underlying tests. Software package size limitations
are therefore quickly exceeded.

In this paper we propose supervised Machine Learning Algorithms to
approximate the p-values of the meta test by \textcite{Bayerhanck_2012}
which tests for the null of non-cointegration. This approach might
reduce the size of associated software packages considerably. The
algorithms are trained on simulated data for various specifications of
the aforementioned test.

\textcolor{red}{Ergebnis der Models (1-2 Sätze)}

\textcolor{red}{Inhalt Paper}

\hypertarget{bayer-hanck-test}{%
\section{Bayer Hanck Test}\label{bayer-hanck-test}}

The choice as to which of the available cointegration tests to use is a
recurrent issue in econometric time series analysis.
\textcite{Bayerhanck_2012} propose powerful meta tests which provide
unambiguous test decisions. They combine several residual- and
system-based tests in the manner of Fisher's \autocite*{Fisher_1932}
Chi-squared test.

Bayer and Hanck build their paper on previous work from
\textcite{Pesavento_2004}, who defines the underlying model as
\(z'_t = [x'_t, y_t]\), with \(x_t\) being an \(n_1 \times 1\) vector
and \(y_t\) a scalar, which displays the cointegration relation. They
can be written as \textbackslash begin\{subequations\} \label{eq:1}
\begin{align}
\Delta x_t &= \tau_1 + v_{1t} \label{eq:11} \\
y_t &= (\mu_2 - \gamma' \mu_1) + (\tau_2 - \gamma' \tau_1) t + \gamma' x_t + u_t, \label{eq:12} \\
u_t &= \rho u_{t-1} + v_{2t}. \label{eq:13}
\end{align} \textbackslash end\{subequation\} \(\Delta x_t\) presents
the regressor dynamics. \(\mu_1\), \(\mu_2\), \(\tau_1\) and \(\tau_2\)
are the deterministic parts of the model. They are subject to the
following restrictions: (i) \(\mu_2 - \gamma' \mu_1\) and \(\tau = 0\)
which translates to no deterministics, (ii) \(\tau = 0\) which
corresponds to a constant in the cointegrating vector, (iii)
\(\tau_2 - \gamma' \tau_1 = 0\), a constant plus trend.

\(v_t = [v'_{1t} v_{2t}]'\) with \(\Omega\) the long-run covariance
matrix of \(v_t\). For derivation of \(v_t\) see
\textcite{Pesavento_2004}. Pesavento shows that \{\(v_t\)\} satisfies an
FCLT,
i.e.~\(T^{-1/2} \sum^{[T \cdot]}_{t=1} v_t \Rightarrow \Omega^{1/2} W(\cdot)\).
It is further assumed that the \(x_t\) are not cointegrated.

It clearly follows from \eqref{eq:13} that \(z_t\) is cointegrated if
\(\rho < 1\). Hence the null hypothesis of no cointegration is
\(H_0: p = 1\). Furthermore, Pesavento introduces two other parameters.
First, \(\text{R}^2\) measures the squared correlation of \(v_{1t}\) and
\(v_{2t}\). It can be interpreted as the influence of the right-hand
side variables in \eqref{eq:12}. It ranks between zero and one. When
there is no long-run correlation between those variables and the errors
from the cointegration regression, \(\text{R}^2\) equals zero. Secondly,
the number of lags is approximated by a finite number \(k\).

\textcolor{red}{Assumptions (BH S. 84)?}

Bayer and Hanck's \autocite*{Bayerhanck_2012} meta test considers the
test statistics of up to four stand-alone tests. Namely, these are the
tests of \textcite{Englegranger_1987}, \textcite{Johansen_1988},
\textcite{Boswijk_1994} and \textcite{Banerjee_1998}. For the sake of
brevity the detailed derivation of the underlying tests has been
deliberately omitted here.

\textcite{Englegranger_1987} propose a two-step procedure to test the
null hypothesis of no cointegration against the alternative of at least
one cointegrating vector. First, the long-run relationship between
\(y_t\) and \(\mathbf{x}_t\) is estimated by least squares regression.
The obtained residuals \(\hat{u}_t\) are then tested for a unit root.
For this, Engle and Granger suggest the use of the \(t\)-statistic
\(t^{\text{ADF}}_\gamma\) in the Augmented Dickey-Fuller (ADF)
regression: \begin{equation}
\Delta \hat{u}_t = \gamma \hat{u}_{t-1} + \sum^{k}_{i=1} \pi_i \Delta \hat{u}_{t-i} + \varepsilon_t.
\label{eq:2}
\end{equation} The rejection of a unit root points to a cointegration
relationship.

Johansen's \autocite*{Johansen_1988} maximum eigenvalue test is a
system-based test that allows for several cointegration relationships.
Take the vector error correction model (VECM)\footnote{Due to practical
  reasons we omit the derivation of the VECM which is presumed to be
  known.} \begin{equation}
\Delta \mathbf{z}_t = \mathbf{\Pi z}_{t-1} + \sum^{k}_{i = 1} \mathbf{\Gamma}_p \Delta \mathbf{z}_{t-p} + \mathbf{d}_t + \mathbf{\varepsilon}_t.
\label{eq:3}
\end{equation} \textcolor{red}{blabla Johansen test statistic}

\textcolor{red}{Banerjee and Boswijk}

To combine the results from the underlying tests Bayer and Hanck draw
upon Fisher's combined probability test \autocite{Fisher_1932}. It
merges the tests using the formula

\begin{equation}
\tilde{\chi}^2_{\mathcal{I}} := -2 \sum_{i \in \mathcal{I}} \ln{(p_i)}. 
\label{eq:4}
\end{equation}

Let \(t_i\) be the \(i^{th}\) test statistic. If test \(i\) rejects for
large values, take \(\xi_i := t_i\). If test \(i\) rejects for small
values, take \(-\xi_i := t_i\). With
\(\Xi_i(x) := \text{Pr}_{\mathcal{H_0}}(\xi_i \geq x)\) the p-value of
the \(i^{th}\) test is \(p_i := \Xi_i(\xi_i)\).

\textcite{Fisher_1932} shows that under the assumption of independence
the null distribution of \(\tilde{\chi}^2_{\mathcal{I}}\) follows a
chi-squared distribution with \(2\mathcal{I}\) degrees of freedom. If
this assumption is violated the null distribution is less evident. Here,
the latter case occurs, as the \(\xi_i\) are not independent. The
\(\tilde{\chi}^2_{\mathcal{I}}\), however, have well-defined asymptotic
null distributions \(F_{\mathcal{F_I}}\), as
\(\tilde{\chi}^2_{\mathcal{I}} \rightarrow_d \mathcal{F_I}\) under
\(\mathcal{H}_0\) if \(T \rightarrow \infty\), with \(\mathcal{F_I}\)
some random variable. It is therefore feasible to simulate the joint
null distribution of the \(\xi_i\) to obtain the distribution
\(F_{\mathcal{F_I}}\) of \eqref{eq:4}. The \(F_{\mathcal{F_I}}\) depend
on which and how many tests are combined. The distributions of the
\(\xi_i\) depend on \(K-1\) and the deterministic case.

\hypertarget{simulation}{%
\section{Simulation}\label{simulation}}

In this section, we describe the simulation of the null distribution of
the Bayer Hanck meta test. The objective is to obtain data for training
machine learning algorithms on approximating the p-values of the
aforementioned test. In consideration of the different forms of the meta
test we generate six data sets. These vary according to the specific
combinations of the underlying tests and also account for the
above-mentioned restrictions on the deterministic parts of the model.

The following approach relies largely on previous work by
\textcite{Pesavento_2004}. For calculating the Bayer Hanck test
statistic we require the p-values of the underlying tests. For this, we
simulate their null distributions. It can be shown that asymptotically
these are functions of standard Brownian motions. Here, the latter are
constructed by step functions using Gaussian random walk with
\(N = 1000\) observations. The number of repetitions is set to
1,000,000. Furthermore, we consider
\(\text{R}^2 \in \{0, 0.05, 0.1, ..., 0.95\}\), \(k = 11\) and
\(c = 0\)\footnote{Since we solely aim at simulating the distribution of
  the null of no cointegration we will not consider any further values
  of \(c\) here.} \textcolor{red}{(c mal definieren)}.

From the mass of test statistics we build the cumulative distribution
function of each underlying test and calculate the respective p-values.
These are inserted into \eqref{eq:4} to eventually obtain the Bayer
Hanck test statistics. Analogous to the previous approach, we deduce the
associated null distribution and the p-values.

\hypertarget{models}{%
\section{Models}\label{models}}

\hypertarget{package}{%
\section{Package}\label{package}}

\pagebreak

\pagenumbering{Roman}
\setcounter{page}{3}
\addcontentsline{toc}{section}{References}
\printbibliography[title = References]
\cleardoublepage

\begin{refsection}
\nocite{R-base}
\nocite{R-stargazer}
\nocite{R-stringr}
\nocite{R-tidyr}
\nocite{R-dplyr}
\nocite{R-glmnet}
\nocite{R-class}
\nocite{R-MASS}
\nocite{R-plm}
\nocite{R-leaps}
\nocite{R-caret}
\nocite{R-tree}
\nocite{R-gbm}
\nocite{R-plotmo}
\nocite{R-pls}
\nocite{R-splines}
\nocite{R-tictoc}
\nocite{R-plotly}
\nocite{R-inspectdf}
\nocite{R-rpart}
\nocite{R-rpart.plot}
\nocite{R-stargazer}
\nocite{R-knitr}
\nocite{R-purrr}
\nocite{R-randomForest}
\nocite{R-rstudioapi}





\nocite{R-Studio}

\printbibliography[title = Software-References]
\addcontentsline{toc}{section}{Software-References}
\end{refsection}

\cleardoublepage
\appendix
\setcounter{table}{0}
\setcounter{figure}{0}
\renewcommand{\thetable}{A\arabic{table}}
\renewcommand{\thefigure}{A\arabic{figure}}

\newgeometry{top = 2.5cm, left = 2.5cm, right = 2cm, bottom = 2cm}

\hypertarget{appendices}{%
\section{Appendices}\label{appendices}}

\FloatBarrier

\begin{table}
    \centering
    \begin{tabular}{rlc}
        Number & Functional form & Range of $\gamma$ \\
        \toprule
        $1$ & $p = \text{poly}(t, \gamma) + \left( 1/k \right)$ & $\gamma \in \mathbb{Z} \left[3, 10 \right]$\\ 
        $2$ & $p = \text{poly}(t, \gamma) + \left( 1/k \right) + \text{poly}(t, \gamma) * 1/k $ & $\gamma \in \mathbb{Z} \left[3, 10 \right]$\\
        $3$ & $p = \text{poly}(t, \gamma) + \log (k) + \text{poly}(k, \gamma) * \log (k) $ & $\gamma \in \mathbb{Z} \left[3, 10 \right]$\\
        $4$ & $p = \text{poly}(t, \gamma) + k + \left( 1/k \right)$ & $\gamma \in \mathbb{Z} \left[3, 10 \right]$\\
        $5$ & $p = \text{poly}(\log (t), \gamma) + \log (k)$ & $\gamma \in \mathbb{Z} \left[3, 10 \right]$\\
        \midrule
        $6$ & $p = \text{poly}(\log (t), \gamma) * \log (k)$ & $\gamma \in \mathbb{Z} \left[3, 10 \right]$\\
        $7$ & $p = \text{poly}(\log (t), \gamma) + k$ & $\gamma \in \mathbb{Z} \left[3, 10 \right]$\\ 
        $8$ & $p = \text{poly}(\log (t), \gamma) * k$ & $\gamma \in \mathbb{Z} \left[3, 10 \right]$\\
        $9$ & $p = \text{poly}(\log (t), \gamma) * k + 1/k $ & $\gamma \in \mathbb{Z} \left[3, 10 \right]$\\    
        $10$ & $p = \text{poly}(\log (t), \gamma) * \log(k)$ & $\gamma \in \mathbb{Z} \left[3, 10 \right]$\\ 
        \midrule
        $11$ & $p = \text{poly}(\log (t), \gamma) * \log(k) + 1/k$ & $\gamma \in \mathbb{Z} \left[3, 10 \right]$\\
        $12$ & $p = \poly (\bc (t), \gamma) * \log(k) + \poly (\bc (t), \gamma) * 1/k$ & $\gamma \in \mathbb{Z} \left[3, 10 \right]$\\
        $13$ & $p = \poly (\bc (t), \gamma) * \log(k) + \poly (\bc (t), \gamma) * \sqrt{k}$ & $\gamma \in \mathbb{Z} \left[3, 10 \right]$\\
        $14$ & $p = \poly (\bc (t), \gamma) * \log(k) + \poly (\bc (t), \gamma) * 1/k +\sqrt{k}$ & $\gamma \in \mathbb{Z} \left[3, 10 \right]$\\
        $15$ & $ \bc (p) = \poly (\bc (t), \gamma) + \log(k)$ & $\gamma \in \mathbb{Z} \left[3, 10 \right]$\\       
        \midrule
        $16$ & $\bc (p) = \poly (\bc (t), \gamma) * \log(k)$ & $\gamma \in \mathbb{Z} \left[3, 10 \right]$\\
        $17$ & $\bc (p) = \poly (\bc (t), \gamma) * \log(k) + 1/k$ & $\gamma \in \mathbb{Z} \left[3, 10 \right]$\\
        $18$ & $\bc (p) = \poly (\bc (t), \gamma) * \log(k) + \poly (\bc (t), \gamma) * 1/k$ & $\gamma \in \mathbb{Z} \left[3, 10 \right]$\\
        $19$ & $\bc (p) = \poly (\bc (t), \gamma) * \log(k) +  \poly (\bc (t), \gamma) * \sqrt{k}$ & $\gamma \in \mathbb{Z} \left[3, 10 \right]$\\
        $20$ & $\log (p) = \poly (\bc (t), \gamma) * \log(k)$ & $\gamma \in \mathbb{Z} \left[3, 10 \right]$\\
        \midrule
        $21$ & $\log (p) = \poly (\bc (t), \gamma) * \log(k) + 1/k$ & $\gamma \in \mathbb{Z} \left[3, 10 \right]$\\
        $22$ & $\log (p) = \poly (\bc (t), \gamma) * \log(k) + \poly (\bc (t), \gamma) * \sqrt{k}$ & $\gamma \in \mathbb{Z} \left[3, 10 \right]$\\
        $23$ & $\log (p) = \poly (\bc (t), \gamma) * \log(k) + \poly (\bc (t), \gamma) * 1/k$ & $\gamma \in \mathbb{Z} \left[3, 10 \right]$\\
        $24$ & $\log (p) = \poly (\bc (t), \gamma) * \log(k) + \poly (\bc (t), \gamma) * 1/k + \sqrt{k}$ & $\gamma \in \mathbb{Z} \left[3, 10 \right]$\\
        \bottomrule
    \end{tabular}
    \caption{Description of all tested models....}
    \label{tab:func_form}    
\end{table}

\begin{table}
\resizebox{\textwidth}{!}{%
    \centering
    \begin{tabular}{llccccc}
        & &\multicolumn{2}{c}{Full distribution} &  & \multicolumn{2}{c}{Lower tail $(p < 0.2)$} \\ \cline{3-4} \cline{6-7}
        Functional form & & RMSE & cRMSE & & RMSE & cRMSE\\
        \toprule
        $\bc (p) = \poly(\bc (t), 10) * \log(k) + \poly(\bc (t), 10) * \sqrt{k}$ &  & $4.97 \cdot 10^{-4}$& $4.69 \cdot 10^{-4}$ & & $8.05 \cdot 10^{-4}$ &  $7.16 \cdot 10^{-4}$\\
        $\bc (p) = \poly(\bc (t), 10) * \log(k) + \poly(\bc (t), 10) * 1/k $&  & $5.39 \cdot 10^{-4}$ & $5.11 \cdot 10^{-4}$ & & $ 8.54 \cdot 10^{-4}$ &  $7.61 \cdot 10^{-4}$\\
        $p = \poly(\bc (t), 10) * \log(k) + \poly(\bc (t), 10) * \sqrt{k}$ &  & $7.68 \cdot 10^{-4}$ & $6.91 \cdot 10^{-4}$ & & $1.01 \cdot 10^{-3}$ & $8.97 \cdot 10^{-4}$\\
        $p = \poly(\bc (t), 10) * \log(k) + \poly(\bc (t), 10) * 1/k + \sqrt{k}$ &  & $7.79 \cdot 10^{-4}$ & $7.04 \cdot 10^{-4}$ & & $1.05 \cdot 10^{-3} $ & $9.31 \cdot 10^{-4}$\\
        $p = \poly(\bc (t), 10) * \log(k) + \poly(\bc (t), 10) * 1/k$&  & $7.82 \cdot 10^{-4}$ & $7.07 \cdot 10^{-4}$ & & $1.06 \cdot 10^{-3}$ & $9.41 \cdot 10^{-4}$\\
        \bottomrule
    \end{tabular}
    }
    \caption{The 5 best models for the combined Non-Cointegration test of Bayer and Hanck, where \textit{all} underlying test are included and case 1.}
    \label{tab:ALL_case_1}   
\end{table}

\begin{table}
\resizebox{\textwidth}{!}{%
    \centering
    \begin{tabular}{llccccc}
        & &\multicolumn{2}{c}{Full distribution} &  & \multicolumn{2}{c}{Lower tail $(p < 0.2)$} \\ \cline{3-4} \cline{6-7}
        Functional form & & RMSE & cRMSE & & RMSE & cRMSE\\
        \toprule
        $\log (p) = \poly(\bc (t), 10) * \log(k)$ & & $1.27 \cdot 10^{-3}$& $1.25 \cdot 10^{-3}$ & & $1.05 \cdot 10^{-3}$ &  $9.52 \cdot 10^{-4}$\\
        $\log (p) = \poly(\bc (t), 10) * \log(k) + \poly(\bc (t), 10) * \sqrt{k}$&  & $6.82 \cdot 10^{-4}$ & $6.22 \cdot 10^{-4}$& &$ 1.28 \cdot 10^{-3}$ &  $1.12 \cdot 10^{-3}$\\
        $\log (p) = \poly(\bc (t), 10) * \log(k) + \poly(\bc (t), 10) * 1/k$&  & $7.32 \cdot 10^{-4}$ & $6.63 \cdot 10^{-4}$& &$1.39 \cdot 10^{-3}$ & $1.20 \cdot 10^{-3}$\\
        $\log (p) = \poly(\bc (t), 10) * \log(k) + \poly(\bc (t), 10) * 1/k + \sqrt{k}$&  & $8.38 \cdot 10^{-4}$ & $7.78 \cdot 10^{-4}$& &$1.48 \cdot 10^{-3}$ & $1.31 \cdot 10^{-3}$\\
        $\bc (p) = \poly(\bc (t), 10) * \log(k) + \poly(\bc (t), 10) * 1/k$ & & $9.08 \cdot 10^{-4}$ & $8.42 \cdot 10^{-4}$ & & $1.69 \cdot 10^{-3}$ & $1.50 \cdot 10^{-3}$\\
        \bottomrule
    \end{tabular}
    }
    \caption{The 5 best models for the combined Non-Cointegration test of Bayer and Hanck, where \textit{all} underlying test are included and case 2.}
    \label{tab:ALL_case_2}   
\end{table}

\begin{table}[h!]
\resizebox{\textwidth}{!}{%
    \centering
    \begin{tabular}{llccccc}
        & &\multicolumn{2}{c}{Full distribution} &  & \multicolumn{2}{c}{Lower tail $(p < 0.2)$} \\ \cline{3-4} \cline{6-7}
        Functional form & & RMSE & cRMSE & & RMSE & cRMSE\\
        \toprule
        $\log (p) = \poly(\bc (t), 10) * \log(k) + \poly(\bc (t), 10) * \sqrt{k}$ & & $4.58 \cdot 10^{-4}$& $4.55 \cdot 10^{-4}$ & & $3.37 \cdot 10^{-4}$ & $3.16 \cdot 10^{-4}$\\
        $\log (p) = \poly(\bc (t), 10) * \log(k) + \poly(\bc (t), 10) * 1/k $ & & $5.17 \cdot 10^{-4}$ & $5.14 \cdot 10^{-4}$ & & $ 3.90 \cdot 10^{-4}$ &  $3.73 \cdot 10^{-4}$\\
        $\log (p) = \poly(\bc (t), 10) * \log(k)$ & & $1.04 \cdot 10^{-3}$ & $1.04 \cdot 10^{-3}$ & & $6.76 \cdot 10^{-4}$ & $6.50 \cdot 10^{-4}$\\     
        $\log (p) = \poly(\bc (t), 10) * \log(k) + \poly(\bc (t), 10) * 1/k + \sqrt{k}$ & & $1.18 \cdot 10^{-3}$ & $1.17 \cdot 10^{-3}$ & & $2.06 \cdot 10^{-3}$ & $2.05 \cdot 10^{-3}$\\       
        $\bc (p) = \poly(\bc (t), 10) * \log(k) + \poly(\bc (t), 10) * 1/k$ & & $1.16 \cdot 10^{-3}$ & $1.06 \cdot 10^{-3}$ & & $2.08 \cdot 10^{-3}$ & $1.80 \cdot 10^{-3}$\\
        \bottomrule
    \end{tabular}
    }
    \caption{The 5 best models for the combined Non-Cointegration test of Bayer and Hanck, where \textit{all} underlying test are included and case 3.}
    \label{tab:ALL_case_3}   
\end{table}

\begin{table}[h!]
\resizebox{\textwidth}{!}{%
    \centering
    \begin{tabular}{llccccc}
        & &\multicolumn{2}{c}{Full distribution} &  & \multicolumn{2}{c}{Lower tail $(p < 0.2)$} \\ \cline{3-4} \cline{6-7}
        Functional form & & RMSE & cRMSE & & RMSE & cRMSE\\
        \toprule
        $\log (p) = \poly(\bc (t), 10) * \log(k) + \poly(\bc (t), 10) * 1/k$ & & $4.75 \cdot 10^{-4}$& $4.44 \cdot 10^{-4}$ & & $7.81 \cdot 10^{-4}$ & $6.84 \cdot 10^{-4}$\\
        $\log (p) = \poly(\bc (t), 10) * \log(k)$ & & $6.54 \cdot 10^{-4}$ & $5.87 \cdot 10^{-4}$ & & $ 1.01 \cdot 10^{-3}$ &  $7.81 \cdot 10^{-4}$\\
        $\log (p) = \poly(\bc (t), 10) * \log(k) + \poly(\bc (t), 10) * \sqrt{k}$ & & $7.60 \cdot 10^{-4}$ & $6.13 \cdot 10^{-4}$ & & $1.46 \cdot 10^{-3}$ & $1.06 \cdot 10^{-3}$\\     
        $\log (p) = \poly(\bc (t), 10) * \log(k) + \poly(\bc (t), 10) * 1/k + \sqrt{k}$ & & $7.64 \cdot 10^{-4}$ & $7.45 \cdot 10^{-4}$ & & $1.29 \cdot 10^{-3}$ & $1.23 \cdot 10^{-3}$\\       
        $\bc (p) = \poly(\bc (t), 10) * \log(k) + \poly(\bc (t), 10) * 1/k$ & & $1.01 \cdot 10^{-3}$ & $9.17 \cdot 10^{-4}$ & & $1.89 \cdot 10^{-3}$ & $1.65 \cdot 10^{-3}$\\
        \bottomrule
    \end{tabular}
    }
    \caption{The 5 best models for the combined Non-Cointegration test of Bayer and Hanck, where \textit{EG-J} underlying test are included and case 1.}
    \label{tab:EG_J_case_1}  
\end{table}

\begin{table}[h!]
\resizebox{\textwidth}{!}{%
    \centering
    \begin{tabular}{llccccc}
        & &\multicolumn{2}{c}{Full distribution} &  & \multicolumn{2}{c}{Lower tail $(p < 0.2)$} \\ \cline{3-4} \cline{6-7}
        Functional form & & RMSE & cRMSE & & RMSE & cRMSE\\
        \toprule
        $\log (p) = \poly(\bc (t), 10) * \log(k)$ & & $7.36 \cdot 10^{-4}$ & $7.25 \cdot 10^{-4}$ & & $7.04 \cdot 10^{-4}$ & $6.45 \cdot 10^{-4}$\\
        $\log (p) = \poly(\bc (t), 10) * \log(k) + \poly(\bc (t), 10) * 1/k$ & & $5.53 \cdot 10^{-4}$ & $5.12 \cdot 10^{-4}$ & & $9.75 \cdot 10^{-4}$ &  $8.56 \cdot 10^{-4}$\\
        $\log (p) = \poly(\bc (t), 10) * \log(k) + \poly(\bc (t), 10) * \sqrt{k}$ & & $5.53 \cdot 10^{-4}$ & $5.11 \cdot 10^{-4}$ & & $9.87 \cdot 10^{-4}$ & $8.66 \cdot 10^{-4}$\\     
        $\log (p) = \poly(\bc (t), 10) * \log(k) + \poly(\bc (t), 10) * 1/k + \sqrt{k}$ & & $6.00 \cdot 10^{-4}$ & $5.62 \cdot 10^{-4}$ & & $1.11 \cdot 10^{-3}$ & $1.01 \cdot 10^{-3}$\\       
        $\bc (p) = \poly(\bc (t), 10) * \log(k) + \poly(\bc (t), 10) * \sqrt{k}$ & & $1.05 \cdot 10^{-3}$ & $9.54 \cdot 10^{-4}$ & & $2.00 \cdot 10^{-3}$ & $1.75 \cdot 10^{-3}$\\
        \bottomrule
    \end{tabular}
    }
    \caption{The 5 best models for the combined Non-Cointegration test of Bayer and Hanck, where \textit{EG-J} underlying test are included and case 2.}
    \label{tab:EG_J_case_2}  
\end{table}

\begin{table}[h!]
\resizebox{\textwidth}{!}{%
    \centering
    \begin{tabular}{llccccc}
        & &\multicolumn{2}{c}{Full distribution} &  & \multicolumn{2}{c}{Lower tail $(p < 0.2)$} \\ \cline{3-4} \cline{6-7}
        Functional form & & RMSE & cRMSE & & RMSE & cRMSE\\
        \toprule
        $\log (p) = \poly(\bc (t), 10) * \log(k) + \poly(\bc (t), 10) * \sqrt{k}$ & & $3.85 \cdot 10^{-4}$& $3.73 \cdot 10^{-4}$ & & $5.03 \cdot 10^{-4}$ & $4.58 \cdot 10^{-4}$\\
        $\log (p) = \poly(\bc (t), 10) * \log(k)$ & & $7.55 \cdot 10^{-4}$ & $7.54 \cdot 10^{-4}$ & & $4.85 \cdot 10^{-4}$ & $4.70 \cdot 10^{-4}$\\
        $\log (p) = \poly(\bc (t), 10) * \log(k) + \poly(\bc (t), 10) * 1/k$ & & $3.73 \cdot 10^{-4}$ & $3.59 \cdot 10^{-4}$ & & $5.34 \cdot 10^{-4}$ & $4.83 \cdot 10^{-4}$\\      
        $\log (p) = \poly(\bc (t), 10) * \log(k) + \poly(\bc (t), 10) * 1/k + \sqrt{k}$ & & $4.87 \cdot 10^{-4}$ & $4.76 \cdot 10^{-4}$ & & $8.52 \cdot 10^{-4}$ & $8.19 \cdot 10^{-4}$\\       
        $\bc (p) = \poly(\bc (t), 10) * \log(k) + \poly(\bc (t), 10) * \sqrt{k}$ & & $1.02 \cdot 10^{-3}$ & $9.35 \cdot 10^{-4}$ & & $1.94 \cdot 10^{-3}$ & $1.70 \cdot 10^{-3}$\\
        \bottomrule
    \end{tabular}
    }
    \caption{The 5 best models for the combined Non-Cointegration test of Bayer and Hanck, where \textit{EG-J} underlying test are included and case 3.}
    \label{tab:EG_J_case_3}  
\end{table}

\restoregeometry

\cleardoublepage
\newpage
\renewcommand*{\mkbibnamefamily}[1]{\textbf{#1}}
\renewcommand*{\mkbibnamegiven}[1]{\textbf{#1}}
\renewcommand*{\mkbibnameprefix}[1]{\textbf{#1}}
\renewcommand*{\mkbibnamesuffix}[1]{\textbf{#1}}


% \printbibliography[title=References]
%\pagenumbering{arabic}


\newpage
\textbf{Eidesstattliche Versicherung}

\bigskip

Ich versichere an Eides statt durch meine Unterschrift, dass ich die vorstehende Arbeit selbständig und ohne fremde Hilfe angefertigt und alle Stellen, die ich wörtlich oder annähernd wörtlich aus Veröffentlichungen entnommen habe, als solche kenntlich gemacht habe, mich auch keiner anderen als der angegebenen Literatur oder sonstiger Hilfsmittel bedient habe. Die Arbeit hat in dieser oder ähnlicher Form noch keiner anderen Prüfungsbehörde vorgelegen.

\vspace{1cm}
\rule{0pt}{2\baselineskip} %
\par\noindent\makebox[2.25in]{\indent Essen, den \hrulefill} \hfill\makebox[2.25in]{\hrulefill}%
\par\noindent\makebox[2.25in][l]{} \hfill\makebox[2.25in][c]{Jens Klenke and Janine Langerbein}%


\end{document}
